\addcontentsline{toc}{chapter}{СПИСОК ИСПОЛЬЗОВАННЫХ ИСТОЧНИКОВ}
\begin{thebibliography}{}
	\bibitem{intro:1}
	\textit{Текст в эпоху Больших Данных} [Электронный ресурс]. --
	Режим доступа: https://www.osp.ru/os/2012/06/13017063 (дата обращения 26.01.2025)
	\bibitem{intro:2}
	\textit{Инструмент автоматизированного сбора данных для улучшения бизнес-процессов} [Электронный ресурс]. --
	Режим доступа: https://www.tadviser.ru/index.php/Новости:Использование\_инструмента\_автоматизированного\_сбора\_данных\_для\_улучшения\_бизнес-процессов (дата обращения 26.01.2025)
	\bibitem{intro:3}
	\textit{Как искусственный интеллект позволяет упростить рутинные операции} [Электронный ресурс]. --
	Режим доступа: https://ritg.ru/blog/kak\_iskusstvennyy\_intellekt\_pozvolyaet\_uprostit\_rutinnye\_operatsii/ (дата обращения 26.01.2025)
	\bibitem{anal:basic}
	https://psv4.userapi.com/s/v1/d/D1COZdEJlsuolmupPWZhRkBYVgA13Blr-CeWXkWLS6XZ2wd7wGDNzTnIFjxMgG1yfmLw8Ez78AoKS9O_SzmrqyYtiUSRbl5f1-f92sRAJvXZo_L3Rkl97Q/Glubokoe_obuchenie_-_Gudfellou_Ya__2018.pdf
	\bibitem{lib:attention_is_all_you_need}
	https://arxiv.org/pdf/1706.03762v3
	\bibitem{lib:chatgpt}
	https://chatgpt.com/



	% \bibitem{lib:rl}
	% Саттон Р. С. Обучение с подкреплением / Р. С. Саттон, Э. Г. Барто; пер. с англ. – Москва : Бином. Лаборатория знаний, 2017. – 400 с.
	% \bibitem{lib:rlmethods}
	% Кэблинг Л. П. Обучение с подкреплением: обзор / Л. П. Кэблинг, М. Л. Литтман, Э. У. Мур // Journal of artificial intelligence research. – 1996. – Т. 4. – С. 237–285.
	% \bibitem{lib:gymnasium_robotics}
	% \textit{Gymnasium Robotics Documentation} [Электронный ресурс]. --
	% Режим доступа: https://robotics.farama.org/ (дата обращения 26.01.2025)
	% \bibitem{lib:pycharm}
	% \textit{Pycharm 2023 Community edition - среда разработки} [Электронный ресурс]. -- 
	% Режим доступа: https://www.jetbrains.com/ru-ru/pycharm/download/other.html (дата обращения 26.01.2025)

\end{thebibliography}