\ssr{ВВЕДЕНИЕ}

Современный этап цифровой трансформации характеризуется экспоненциальным ростом объёмов неструктурированных текстовых данных — 
от новостных лент и юридических документов до пользовательских отзывов и публикаций в социальных медиа \cite{intro:1}. 
Такой информационный поток создаёт критическую потребность в инструментах автоматического анализа, 
способных идентифицировать ключевые сущности (акторы), действия и временные метки. 
Актуальность разработки программных решений на основе искусственного интеллекта (ИИ) обусловлена 
их возможностью трансформировать рутинные процессы обработки данных в масштабируемые и интеллектуальные системы, 
отвечающие современным требованиям бизнеса и государственных структур.

Во-первых, автоматизация извлечения информации стала неотъемлемым элементом цифровой экономики. 
В сфере медиаанализа системы на базе ИИ позволяют в режиме реального времени отслеживать нарративы и выявлять тренды общественного мнения. 
В области кибербезопасности алгоритмы распознавания подозрительных активностей в текстовых логах способствуют предотвращению атак ещё до их реализации. 
Юридические компании активно применяют подобные технологии для анализа тысяч судебных прецедентов, что позволяет сократить время подготовки к делам на 40-60\% \cite{intro:2}.

Во-вторых, объёмы текстовых данных ежегодно увеличиваются на 55-60\%, а их ручная обработка становится экономически нецелесообразной. 
Например, крупные корпорации могут тратить до 30\% рабочего времени сотрудников на поиск и структурирование информации в документах \cite{intro:3}.
Эти тенденции подчёркивают необходимость внедрения технологий машинного обучения (МО), способных обрабатывать петабайты данных с минимальными затратами человеческих ресурсов.

Целью данной курсовой работы является разработка программного решения на основе искусственного интеллекта, 
способного эффективно определять акторов, действия и временные характеристики их осуществления в текстах с использованием возможностей больших языковых моделей (LLM).
Для достижения этой цели предполагается решить следующие задачи:
\begin{enumerate}
    \item провести анализ существующих методов автоматического извлечения информации из текстов;
    \item изучить возможности и ограничения больших языковых моделей в контексте извлечения семантических компонентов текстовых данных;
    \item разработать прототип программного решения, реализующего алгоритмы идентификации акторов, действий и временных меток на основе выбранной модели LLM;
    \item провести экспериментальную оценку разработанного решения на различных наборах данных для определения его точности и эффективности;
    \item подвести итоги исследования и сформулировать рекомендации по дальнейшему совершенствованию методов.
\end{enumerate}

\clearpage
